
\documentclass[12pt, letterpaper, titlepage]{article}
\usepackage[left=3.5cm, right=2.5cm, top=2.5cm, bottom=2.5cm]{geometry}
\usepackage[MeX]{polski}
\usepackage[utf8]{inputenc}
\usepackage{graphicx}
\usepackage{enumerate}
\usepackage{amsmath} %pakiet matematyczny
\usepackage{amssymb} %pakiet dodatkowych symboli
\title{Kącik kulinarny}
\author{Magda Gessler}
\date{Przepisy z lat 2010-2020}
\begin{document}
\maketitle


\newpage
\section{Przepisy:}
\subsection{Desery:}
\subsubsection{Naleśniki:}
\begin{enumerate}[a.]
\item 1 szklanka mąki pszennej
\item 2 jajka
\item 1 szklanka mleka
\item 3/4 szklanki wody (najlepiej gazowanej)
\item szczypta soli
\item 3 łyżki masła lub oleju roślinnego
\end{enumerate}
\underline{PRZYGOTOWANIE:} \\
Mąkę wsypać do miski, dodać jajka, mleko, wodę i sól. Zmiksować na gładkie ciasto. Dodać roztopione masło lub olej roślinny i razem zmiksować (lub wykorzystać tłuszcz do smarowania patelni przed smażeniem każdego naleśnika).\\
Naleśniki smażyć na dobrze rozgrzanej patelni z cienkim dnem np. naleśnikowej. Przewrócić na drugą stronę gdy spód naleśnika będzie już ładnie zrumieniony i ścięty.
\underline{WSKAZÓWKI:}\\
Do naleśników deserowych można dodać 1 łyżkę cukru.\\
\\{\fontsize{30}{20}\selectfont\textit{SMACZNEGO!!!}\par

\end{document}

