
\documentclass[12pt, letterpaper, titlepage]{article}
\usepackage[left=3.5cm, right=2.5cm, top=2.5cm, bottom=2.5cm]{geometry}
\usepackage[MeX]{polski}
\usepackage[utf8]{inputenc}
\usepackage{graphicx}
\usepackage{enumerate}
\usepackage{amsmath} %pakiet matematyczny
\usepackage{amssymb} %pakiet dodatkowych symboli
\title{Kącik kulinarny}
\author{Magda Gessler}
\date{Przepisy z lat 2010-2020}
\begin{document}
\maketitle


\newpage
\section{Przepisy:}
\subsection{Desery:}
\subsubsection{Naleśniki:}
\begin{enumerate}[a.]
\item 1 szklanka mąki pszennej
\item 2 jajka
\item 1 szklanka mleka
\item 3/4 szklanki wody (najlepiej gazowanej)
\item szczypta soli
\item 3 łyżki masła lub oleju roślinnego
\end{enumerate}
\underline{PRZYGOTOWANIE:} \\
Mąkę wsypać do miski, dodać jajka, mleko, wodę i sól. Zmiksować na gładkie ciasto. Dodać roztopione masło lub olej roślinny i razem zmiksować (lub wykorzystać tłuszcz do smarowania patelni przed smażeniem każdego naleśnika).\\
Naleśniki smażyć na dobrze rozgrzanej patelni z cienkim dnem np. naleśnikowej. Przewrócić na drugą stronę gdy spód naleśnika będzie już ładnie zrumieniony i ścięty.
\underline{WSKAZÓWKI:}\\
Do naleśników deserowych można dodać 1 łyżkę cukru.\\
\\{\fontsize{30}{20}\selectfont\textit{  SMACZNEGO!!!}\par

\newpage
\section{Przepisy-oceny:}
\subsection{tabela:}
\begin{table}[h]
\centering\caption{oceny przepisów}
\begin{tabular}{|c|c|c|}
\hline
Ocena  & średnia & liczba ogółem\\
\hline
bdb & 1298 & 665\\
\hline 
bd & 2348 & 544\\
\hline
średnie & 1298 & 665\\
\hline
słabe & 18 & 76\\
\hline
bardzo słabe & 5 & 44\\
\hline

\end{tabular}
\end{table}

\section{Ćwiczenie 3:}
\subsection{Pacjenci:}
\begin{table}[h]
\centering\caption{System decyzyjny (U,A,d), modelujący problem diagnozy medycznej, której efektem jest decyzja o wykonaniu lub nie wykonaniu operacji wycięcia wyrostka robaczkowego, U={u1,12,...u10}, A={a1,a2}, dcD={TAK/NIE}}
\begin{tabular}{c|c c c}
\hline
\hline
Pacjent  & Ból brzucha & Temperatura ciała& Operacja\\
\hline
u1 & Mocny & Wysoka & Tak\\
u2 & Średni & Wysoka & Tak\\
u3 & Mocny & Średnia & Tak\\
u4 & Mocny & Wysoka & Tak\\
u5 & Mocny & Niska & Tak\\
u6 & Mocny & Wysoka & Tak\\
u7 & Mały & Średnia & Tak\\
u8 & Mały & Wysoka & Tak\\
u9 & Mocny & Niska & Tak\\
u10 & Mały & Wysoka & Tak\\
\hline
\hline
\end{tabular}
\end{table}
\end{document}



